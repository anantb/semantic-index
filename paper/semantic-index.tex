\documentclass[10pt]{article} 
\usepackage{csw} 
\usepackage{cswapa} 
\usepackage{times}
\usepackage{spverbatim}
\usepackage{multicol}

\begin{document} 
\cswtitleblock{
\cswtitle{Semantic Indexing}
\cswauthor{Anant Bhardwaj}{anantb@csail.mit.edu}
\cswaddress{}
}

\section{Introduction}
Majority of information retrieval and web search use the inverted index as the backbone for their keyword based search. Unfortunately, inverted index fail to capture the semantic of the language and thus keyword based approach can only go as far as giving the relevant hits. I propose semantic indexing scheme which can encode natural languages without losing semantic relationships.
 \\
 \\
The encoding of natural language would involve parsing of each sentence, deriving its semantic model and storing the encoded representation. In the below section I discuss the appropriate schema to store the derived semantic model, such that semantic information can be recovered accurately. 

\section{Schema}
\begin{multicols}{2}

\begin{tabular}{|p{0.95\linewidth}|} 
\hline                   
\textbf{Events}
\\
\hline
id\\
\hline
\end{tabular}



\begin{tabular}{|p{0.95\linewidth}|} 
\hline                   
\textbf{Actions}
\\
\hline
id\\
\hline
action\\
\hline
\end{tabular}



\begin{tabular}{|p{0.95\linewidth}|} 
\hline                   
\textbf{Objects}
\\
\hline
id\\
\hline
object\\
\hline
\end{tabular}




\begin{tabular}{|p{0.95\linewidth}|} 
\hline                   
\textbf{Adjectives}
\\
\hline
id\\
\hline
adjective\\
\hline
\end{tabular}





\begin{tabular}{|p{0.95\linewidth}|} 
\hline                   
\textbf{Adverbs}
\\
\hline
id\\
\hline
adverb\\
\hline
\end{tabular}



\begin{tabular}{|p{0.95\linewidth}|} 
\hline                   
\textbf{Tenses}
\\
\hline
id\\
\hline
tense\\
\hline
\end{tabular}



\begin{tabular}{|p{0.95\linewidth}|} 
\hline                   
\textbf{Event-Actions}
\\
\hline
id\\
\hline
event\_id\\
\hline
action\_id\\
\hline
tense\_id\\
\hline
\end{tabular}




\begin{tabular}{|p{0.95\linewidth}|} 
\hline                   
\textbf{Event-Agents}
\\
\hline
id\\
\hline
event\_id\\
\hline
agent\_id(object\_id)\\
\hline
\end{tabular}
\label{table:nonlin}




\begin{tabular}{|p{0.95\linewidth}|} 
\hline                   
\textbf{Event-Patients}
\\
\hline
id\\
\hline
event\_id\\
\hline
patient\_id(object\_id)\\
\hline
\end{tabular}




\begin{tabular}{|p{0.95\linewidth}|} 
\hline                   
\textbf{Event-Instruments}
\\
\hline
id\\
\hline
event\_id\\
\hline
instrument\_id(object\_id)\\
\hline
\end{tabular}



\begin{tabular}{|p{0.95\linewidth}|} 
\hline                   
\textbf{Event-Beneficiaries}
\\
\hline
id\\
\hline
event\_id\\
\hline
beneficiary\_id(object\_id)\\
\hline
\end{tabular}


\begin{tabular}{|p{0.95\linewidth}|} 
\hline                   
\textbf{Event-Locations}
\\
\hline
id\\
\hline
event\_id\\
\hline
location\_id(object\_id)\\
\hline
\end{tabular}



\begin{tabular}{|p{0.95\linewidth}|} 
\hline                   
\textbf{Event-Adjectives}
\\
\hline
id\\
\hline
event\_id\\
\hline
object\_id\\
\hline
adjective\_id\\
\hline
\end{tabular}



\begin{tabular}{|p{0.95\linewidth}|} 
\hline                   
\textbf{Event-Adverbs}
\\
\hline
id\\
\hline
event\_id\\
\hline
action\_id\\
\hline
adverb\_id\\
\hline
\end{tabular}

\end{multicols}

\section{Sample Example}









\bibliography{references} \bibliographystyle{cswapa}
\end{document} 
